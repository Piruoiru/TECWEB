\section{Accessibilità}
\subsection{Separazione tra contenuto, presentazione e struttura}
Il sito è stato organizzato in tre sezioni principali per gestire al meglio gli accessi delle diverse categorie di utenti. La struttura dei contenuti è stata realizzata utilizzando HTML5, sfruttando i tag semantici e le nuove funzionalità offerte dallo standard. Il layout e la presentazione sono stati definiti con CSS3, mentre il comportamento dinamico è stato gestito tramite Javascript. Durante lo sviluppo, è stato verificato il rispetto degli standard del W3C utilizzando strumenti come i validatori di HTML e CSS.
\subsection{Attributi ARIA}
\textbf{\textit{TODO}}
\subsection{Link}
I link all'interno del sito sono ben segnalati:
\begin{itemize}
    \item Sono colorati di bianco su sfondo blu se non sono ancora stati visitati.
    \item Se le pagine sono gia state visitate i link vengono sottolineati con il colore giallo.
    \item Mentre se si fa hover su di essi, lo sfondo diventa giallo e la scritta di colore nero.
\end{itemize}
\subsection{Immagini}
Le immagini del sito sono state ottimizzate utilizzando formati come PNG, garantendo un equilibrio ideale tra qualità e leggerezza grazie a tecniche di compressione e ridimensionamento. Ogni immagine include un attributo alt, ad eccezione di quelle decorative, per migliorare l’accessibilità. Le immagini sono state scelte principalmente per rappresentare le attrazioni e gli spettacoli offerti nel  parco divertimenti, valorizzando l’esperienza visiva degli utenti. Il rispetto degli standard W3C è stato assicurato attraverso l’uso di strumenti di validazione dedicati.
\subsection{Contrasto Colori}
La scelta dei colori per il sito web è stata il risultato di uno studio accurato e approfondito, con particolare attenzione al rispetto degli standard di accessibilità WCAG a livello AA. Numerose combinazioni sono state scartate poiché non garantivano un contrasto adeguato. 
Alcuni esempi di colori presi in considerazione per la progettazione del sito sono i seguenti:  
\begin{itemize}
    \item Bianco: \#FFF
    \item Blu: \#005AB5
    \item Giallo: \#FFD700
    \item Rosso: \#AA0000
    \item Azzurro: \#D1FFFF
\end{itemize}
Tuttavia, alcune combinazioni cromatiche sono state scartate poiché non rispettavano il livello AA di contrasto secondo gli standard WCAG, non raggiungendo un rapporto minimo di 3.0.  
Ad esempio: 
\begin{itemize}
    \item Il contrasto tra il colore \#FFD700 (Giallo) e il colore \#D1FFFF (Azzurro) era pari a 1.2, risultato non sufficiente.
    \item Il contrasto tra il colore \#AA0000 (Rosso) e il colore \#005AB5 (Blu) era pari a 1.15, anch'esso non conforme.
\end{itemize}
Queste analisi hanno guidato la scelta dei colori finali per garantire un'esperienza utente accessibile e inclusiva.
\\
Dopo un'analisi meticolosa, sono stati selezionati tre colori principali, oltre al bianco (\#FFF) e al nero (\#000):
\begin{itemize}
    \item Colore primario: \#000a99
    \item Colore secondario: \#FFC300
    \item Colore terziario: \#0071EB
    \item Colore quaternario: \#BD0000
\end{itemize}
Tutti i colori sono stati verificati utilizzando strumenti come il Color Contrast Checker di AudioEye, assicurandosi che il contrasto tra di essi raggiungesse almeno il livello AA, garantendo così un’esperienza visiva inclusiva e accessibile per tutti gli utenti.
\subsection{Focus}
La gestione del focus è stata un elemento centrale nello sviluppo del sito web, garantendo un’esperienza inclusiva e accessibile per un ampio numero di utenti. Grazie all’utilizzo di link ben strutturati e del tag tabindex, è stato possibile rendere tutti i link e i pulsanti facilmente raggiungibili tramite il tasto Tab. Questo approccio assicura una navigazione fluida anche per gli utenti che utilizzano solo la tastiera, migliorando significativamente l’accessibilità del sito.
\subsection{Strumenti di testing}
\textbf{\textit{TODO}}
\subsection{Ambienti di test}
Il sito e le sue funzionalità sono state testate su ambiente Linux, Windows e MacOS, sui seguenti browser: \textbf{\textit{TODO}}
\begin{itemize}
    \item \textbf{Chrome:}
        \begin{itemize}
            \item Versione del sito perfettamente funzionante
            \item Versione di Chrome aggiornata: 132.0.6834.111
        \end{itemize}
    \item \textbf{Firefox:}
        \begin{itemize}
            \item Versione del sito perfettamente funzionante
            \item Versione di Firefox aggiornata: 134.0.2
        \end{itemize}
    \item \textbf{Edge:} 
        \begin{itemize}
            \item Versione del sito perfettamente funzionante
            \item Versione di Edge aggiornata: 132.0.2957.127
        \end{itemize}
    \item \textbf{Safari:} 
        \begin{itemize}
            \item Versione del sito perfettamente funzionante
            \item Versione di Safari aggiornata: 18.3
        \end{itemize}
\end{itemize}
\subsection{Dispositivi}
Il sito e le sue funzionalit\`a sono stati sottoposti a test approfonditi su diverse piattaforme, tra cui PC desktop, smartphone e tablet. Oltre a sfruttare gli strumenti per sviluppatori integrati nei vari browser, sono state effettuate verifiche dirette su dispositivi reali, come smartphone e tablet, per garantire un'esperienza utente ottimale e un'usabilità impeccabile su ogni tipo di schermo.
\subsection{Risultati di Lighthouse}
Lighthouse è uno strumento integrato nel browser Chrome che consente di analizzare le prestazioni e la qualità complessiva di un sito web. I valori riportati rappresentano i seguenti aspetti:  
\begin{itemize}
    \item \textbf{Prestazioni}: Indica il tempo di caricamento delle risorse e la reattività del sito, misurando la velocità complessiva della pagina.
    \item \textbf{Accessibilità}: Valuta quanto il sito è utilizzabile per persone con disabilità, basandosi su pratiche e standard internazionali.
    \item \textbf{Best Practice}: Analizza la conformità del sito a standard di sicurezza e sviluppo moderni.
    \item \textbf{SEO}: Misura l'ottimizzazione del sito per i motori di ricerca, valutando fattori come tag, struttura e metadati.
\end{itemize}

Sono state verificate tutte le pagine del sito utilizzando Lighthouse. I risultati sono suddivisi tra modalità \textbf{standard} e \textbf{mobile} per ogni pagina.

\begin{table}[H]
    \hspace{-2.2cm}
    \renewcommand{\arraystretch}{1.3} % Per aumentare lo spazio tra le righe
    \setlength{\tabcolsep}{10pt} % Per aumentare lo spazio tra le colonne
    \begin{tabular}{|c|c|c|c|c|c|}
        \hline
        \textbf{Pagina} & \textbf{Dispositivo} & \textbf{Prestazioni} & \textbf{Accessibilità} & \textbf{Best Practice} & \textbf{SEO}  \\ 
        \hline
        Homepage & Standard & 100 & 100 & 100 & 100 \\ 
        \hline
        \rowcolor[gray]{0.9}
        Homepage & Mobile & 100 & 100 & 100 & 100 \\ 
        \hline
        Scopri le Attrazioni & Standard & 100 & 100 & 100 & 100 \\ 
        \hline
        \rowcolor[gray]{0.9}
        Scopri le Attrazioni & Mobile & 100 & 100 & 100 & 100 \\ 
        \hline
        I nostri shows & Standard & 100 & 100 & 100 & 100 \\ 
        \hline
        \rowcolor[gray]{0.9}
        I nostri shows & Mobile & 100 & 100 & 100 & 100 \\ 
        \hline
        Orari di apertura & Standard & 100 & 100 & 100 & 100 \\ 
        \hline
        \rowcolor[gray]{0.9}
        Orari di apertura & Mobile & 100 & 100 & 100 & 100 \\ 
        \hline
        Acquista i biglietti & Standard & 100 & 100 & 100 & 100 \\ 
        \hline
        \rowcolor[gray]{0.9}
        Acquista i biglietti & Mobile & 100 & 100 & 100 & 100 \\ 
        \hline
    \end{tabular}
    \caption{Valutazione delle prestazioni, accessibilità e ottimizzazione SEO per le diverse pagine del sito.}
    \label{tab:prestazioni}
\end{table}
