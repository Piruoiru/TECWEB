\section{Progettazione}

\subsection{Progettazione del sito}
Il layout delle pagine principali del sito sono state progettate utilizzando il software FIGMA. Questo ci ha permesso di realizzare una prima versione delle pagine, facilitando l'organizzazione dei contenuti e la definizione dello stile grafico. Inoltre, l'uso di FIGMA ha consentito di considerare fin dalle prime fasi del progetto gli aspetti legati all'accessibilità come la scelta dei colori, garantendo un'esperienza utente inclusiva e conforme agli standard di usabilità.

\subsection{Tipi di utente}
\begin{itemize}
    \item \textbf{Utente non autenticato:} è la principale tipologia di utente. Può visualizzare gli show, le attrazioni, gli orari e la mappa. Inoltre, può registrarsi in caso non sia già in possesso di un account o accedere in caso lo sia già.
    \item \textbf{Utente autenticato:} oltre alle funzionalità disponibili per l'utente non autenticato potrà acquistare biglietti, visualizzare lo storico degli acquisti, modificare il proprio profilo o eliminarlo.
    \item \textbf{Amministratore:} oltre alle funzionalità disponibili per l'utente registrato potrà creare, modificare o rimuovere gli show. Inoltre, non può eliminare il proprio profilo.
\end{itemize}

\subsection{Struttura del sito}
\subsubsection{Header e navigazione}
L’header del sito è stato progettato per essere il più funzionale possibile in base alle dimensioni dello schermo. Al suo interno si trovano il logo, sempre visibile e una navbar responsive.
\begin{itemize}
    \item Su schermi di piccole dimensioni, la navbar si trasforma in un pulsante che consente di aprire un menu a schermo intero.
    \item Su schermi più grandi, invece, la navbar mostra direttamente tutti i collegamenti alle pagine, garantendo una navigazione immediata e intuitiva.
\end{itemize}
La navbar contiene 5 link nella versione per schermi grandi, rispettando il principio di usabilità che consiglia di non superare i 7 collegamenti per garantire un'esperienza di navigazione chiara ed evitare il sovraccarico cognitivo per l'utente. I link presenti in tale navbar sono:
\begin{itemize}
    \item Home
    \item Scopri le attrazioni
    \item I nostri show
    \item Orario di apertura
    \item Acquista biglietti
\end{itemize}
La versione per gli schermi piccoli contiene dei link in più, portando il numero massimo dei link a 8. Per ridurre il sovraccarico cognitivo dell'utente dovuto alla presenza di tanti link è stata inserita una barra orizzontale (hr) che separa i link sempre presenti (anche nella versione per grandi schermi) da quelli aggiunti solo nella navbar mobile. Questi ultimi riguardano solo le operazioni collegate al profilo o all'autenticazione e sono:
\begin{itemize}
    \item Per l'utente autenticato:
    \begin{itemize}
        \item Carrello
        \item Il mio profilo
        \item Logout
    \end{itemize}
    \item Per l'utente non autenticato:
    \begin{itemize}
        \item Accedi o registrati
    \end{itemize}
\end{itemize}
In entrambe le versioni di layout vengono visualizzate le breadcrumb. I link delle breadcrumb sono riconoscibili dal colore e dalla sottilineatura, conformemente alle norme WCAG. A fianco delle breadcrumb sono presenti, solo nella versione per schermi grandi, l'orario di apertura del giorno e alcuni pulsanti utili per le operazioni connesse al profilo. Tali pulsanti permettono di accedere alle funzionalità che nella versione mobile vengono spostate nella navbar principale. In dettaglio sono:
\begin{itemize}
    \item Per l'utente autenticato:
    \begin{itemize}
        \item Carrello
        \item Il mio profilo
        \item Logout
    \end{itemize}
    \item Per l'utente non autenticato:
    \begin{itemize}
        \item Accedi
        \item Registrati
    \end{itemize}
\end{itemize}
Per quanto riguarda l'accessibilità dei link, si rimanda alla sezione dedicata.
\subsubsection{Contenuto}
Per ogni pagina del sito, è stata adottata una struttura semplice e chiara, pensata per permettere all'utente di rispondere facilmente e rapidamente alle tre domande fondamentali:
\begin{itemize}
    \item \textbf{Dove sono?} Grazie alle breadcrumb e all'header sempre visibile in cima alla pagina, l'utente può identificare immediatamente la propria posizione nel sito.
    \item \textbf{Di cosa si tratta?} L'uso appropriato dei tag semantici rende il contenuto principale facilmente riconoscibile, consentendo all'utente di focalizzarsi rapidamente sulle informazioni più rilevanti.
    \item \textbf{Dove posso andare?} Lo schema organizzativo del sito è ampio e poco profondo e permette all'utente di esplorare il sito senza difficoltà.
\end{itemize}
Lo schema organizzativo è ibrido: principalmente è per argomento e in alcuni parti è per task.\\
Per quanto riguarda il contenuto delle pagine principali:
\begin{itemize}
    \item \textbf{Home:} offre una panoramica dei contenuti principali del sito cercando di invitare l'utente ad esplorare altre pagine.
    \item \textbf{Scopri le attrazioni e mappa del parco:} includono la mappa del parco e l'elenco delle attrazioni, con la possibilità di filtrarle per categoria.
    \item \textbf{I nostri show:} presenta gli spettacoli offerti da DreamVille e consente all'amministratore di accedere alle funzionalità di aggiunta, modifica ed eliminazione degli spettacoli.
    \item \textbf{Orari di apertura:} mostra il calendario con gli orari di apertura del parco.
    \item \textbf{Acquista i biglietti:} permette di aggiungere al carrello i biglietti, scegliendo tra intero e ridotto.
    \item \textbf{Carrello:} permette di acquistare i biglietti nel carrello.
    \item \textbf{Il mio profilo e acquisti:} consentono all'utente di visualizzare e gestire i propri dati personali e gli acquisti effettuati.
    \item \textbf{Login:} include un form per accedere al sito.
    \item \textbf{Registrati:} offre un form per creare un nuovo account.
    \item \textbf{Logout:} permette di uscire dal proprio profilo.
\end{itemize}
Sono state inoltre previste delle pagine di errore, pronte ad intervenire, rassicurare e guidare l'utente qualora ci fosse qualche problema.
Le pagine sono:
\begin{itemize}
    \item \textbf{ERROR 401:} La pagina 401 è stata implementata per gestire un utente che ha correttamente eseguito l'accesso al sito ma non possiede le autorizzazioni di accerdervi.
    \item \textbf{ERROR 404:} La pagina 404 è stata implementata per gestire le richieste a risorse non esistenti, rassicurando l'esperienza utente con un messaggio chiaro ma divertente.
    \item \textbf{ERROR 500:} La pagina 500 è stata creata per informare l'utente di errori interni del server, garantendo trasparenza e minimizzando la frustrazione.
\end{itemize}
\subsubsection{Footer}
Nel footer del sito sono presenti gli sponsor di Dreamville. Inoltre, sono state inserite le icone dei principali social media: sebbene non siano link attivi, poiché non conducono a pagine social reali, simboleggiano l'importanza dei social come uno degli strumenti principali per condividere informazioni e notizie con gli utenti.

