\section{Analisi}
\subsection{Descrizione testuale}

Dreamville è un parco divertimenti pensato per far vivere emozioni uniche a tutte le età.
Gli utenti del sito possono svolgere alcune azioni tra le quali:
\begin{itemize}
    \item \textbf{Acquistare un biglietto:} i biglietti interi e ridotti possono essere acquistati presso il sito web.
    \item \textbf{Esplorare la mappa e le attrazioni:} la mappa del parco e i dettagli delle attrazioni sono disponibili presso il sito in modo che gli utenti possano conoscere il parco prima di recarvisi.
    \item \textbf{Calendario del parco:} il calendario aggiornato mostra i giorni e gli orari di apertura.
    \item \textbf{Spettacoli e show:} ogni giorno il parco offre spettacoli dal vivo. La lista degli spettacoli di quel giorno può essere visualizzata nell'apposita sezione.
\end{itemize}
\subsection{Analisi del target di utenza}
Il sito di Dreamville è pensato per qualsiasi età; famiglie, coppie e gruppi di amici che cercano esperienze di divertimento uniche. È perfetto per chi desidera pianificare una giornata indimenticabile e fornisce informazioni su biglietti, attrazioni e spettacoli. È adatto sia a chi visita per la prima volta, grazie alle semplicità di navigazione, sia ai frequentatori abituali che vogliono scoprire eventi e novità.
\\ \\
Analizzando l'utenza con la metafora della pesca:
\begin{itemize}
    \item \textbf{Tiro Perfetto:} l'utente ha un obiettivo chiaro e preciso, sa esattamente cosa cercare e raggiunge direttamente le pagine di interesse per ottenere le informazioni desiderate. Per soddisfare le esigenze di questo tipo di utente, il sito offre già nella navbar i collegamenti alla maggioranza delle pagine del sito. Ogni pagina è inoltre correlata a un solo argomento per facilitare la ricerca delle informazioni.
    
    \item \textbf{Trappola per aragoste:} la struttura del sito guida l'utente nella scoperta dei contenuti, permettendogli di esplorare le varie sezioni e scoprire le attività disponibili a Dreamville in modo intuitivo e coinvolgente. In particolare, il contenuto della homepage guida l'utente verso le principali pagine del sito fornendo un'idea generale dei loro contenuti.
    
    \item \textbf{Pesca con la rete:} grazie a una gerarchia semplice e ben organizzata, la navigazione risulta fluida, consentendo all'utente di muoversi tra le pagine senza difficoltà in modo da prevenirne il sovraccarico cognitivo. In particolare, nelle pagine che lo permettevano sono stati inseriti dei link alle pagine correlate.
    
    \item \textbf{Boa di segnalazione:} l'utente può facilmente distinguere le pagine già visitate dallo stile dei link. Inoltre, in caso acceda nuovamente a una pagina chiusa senza volerlo, il suo profilo viene caricato automaticamente, migliorando l’esperienza d’uso e la continuità della navigazione.
\end{itemize}