\section{Search Engine Optimization}
In questa sezione sono elencati gli accorgimenti presi per migliorare il più possibile il posizionamento nelle SERP (Search Engine Response Page).
\subsection{Utilizzo delle keyword nel tag title}
Per il metatag \textbf{title} sono stati utilizzati al massimo 55 caratteri, spazi inclusi. Inoltre sono state inserite parole chiave al loro interno.
\subsection{Meta tag description}
I metatag \textbf{description} sono stati utilizzati un massimo di 145 caratteri spazi inclusi. Inoltre sono state incluse le \textit{call to action} ad esempio: 
\begin{itemize}
    \item Scopri tutti gli spettacoli mozzafiato di DreamVille! L'effetto WOW è assicurato
    \item Acquista un biglietto e assicurati una giornata spaziale a Dreamville.
\end{itemize}
\subsection{Utilizzo corretto delle intestazioni}
Tutti i livelli di \textbf{intestazioni} (h1, ..., h6) sono stati correttamente inseriti rispettando l'ordine e pensando alla struttura del documento e non a come vengono visualizzati di default. Ciò è stato verificato tramite le estensioni WAVE e Silktide.
\subsection{Attributo Alt}
Sono stati introdotte le \textbf{descrizioni} alt alle immagini in maniera corretta mantenendo il limite di 100 caratteri. Le immagini considerate di decorazione non contengono un alt. Le immagini inserite con il tag svg contengono un title se non sono di decorazione.
\subsection{Velocità di caricamento}
Le immagini utilizzate sono state compresse per \textbf{migliorare la velocità} di caricamento.
\subsection{Qualità di scrittura del codice}
Il codice è stato \textbf{correttamente diviso} tra markup strutturale e markup di presentazione. Inoltre, è stato utilizzato il tag \textless strong\textgreater al posto di \textless b\textgreater per migliorare sia il posizionamento nelle SERP sia l'accessibilità.
\subsection{Eliminazione di link errati}
Non sono presenti \textbf{link rotti o circolari}. Ciò è stato verificato con TotalValidator.
\subsection{Alberatura dei menù facilmente scansionabile dai crawler}
Lo \textbf{schema} del nostro sito è ampio e poco profondo, in modo che le pagine siano raggiungibili velocemente da utenti e crawler.
\subsection{Assenza di contenuti duplicati}
Non sono stati inseriti contenuti duplicati.
\subsection{Possibili ricerche degli utenti}
Le possibili ricerche degli utenti che ci hanno guidati nella scelta delle keywords sono:
\begin{itemize}
    \item DreamVille parco divertimenti
    \item Orari DreamVille
    \item Biglietto intero/ridotto DreamVille
    \item Spettacoli di DreamVille
    \item Login DreamVille
\end{itemize}